\subsection*{Диапазоны виртуальной памяти}

Если мы находимся в привилегированном режиме ядра, то ты мы сами должны позаботиться, чтобы \textit{NULL} указывал на никуда неотображаемый адрес.
\\

\textbf{Схема виртуальной памяти ядра:}
\begin{enumerate}
\item Данные ядра
\begin{itemize}
\item NULL
\item Адреса input-output
\item DMA
\item Код ядра
\item Данные
\end{itemize}
\item Куча ядра
\begin{itemize}
\item Организация кучи (<<дерево>>)
\item Диапазон адресов
\end{itemize}
\item Некоторый пустой промежуток
\item Стек ядра

Стек ядра и стек обычной программы не должны пересекаться.
\end{enumerate}

\textit{kmalloc} --- захватывает кусок данных в куче ядра
\\

\textbf{Схема виртуальной памяти обычной программы:}
\begin{enumerate}
\item Ядро
\item Код
\item Данные
\item Куча
\item Пустое пространство
\item Стек
\end{enumerate}

\section{Процесс}

\textbf{Процесс} --- программа, обладающая некоторыми правами на ресурсы.
\\

Контекст процесса:
\begin{enumerate}
\item Аппаратная составляющая (процесс ей пользуется только во время своего выполнения)
\begin{itemize}
\item Регистры
\item Адреса в ОП
\end{itemize}
\item Пользовательская составляющая
\begin{itemize}
\item Код процесса
\item Данные процесса
\end{itemize}
\item Системная составляющая
\begin{itemize}
\item Таблицы страниц
\item Целочисленные значения
\item Ограничения
\item Таблица открытых файлов
\item Список обработчиков сигналов
\item Сигнальная маска и вектор сигналов, очередь сигналов
\item Список IPC-объектов
\item Список аргументов \verb!main!
\end{itemize}
\end{enumerate}

Исполняемый файл (ELF, MZ):
\begin{enumerate}
\item Код
\item Данные
\item Служебная информация
\end{enumerate}


Целочисленные значения:
\begin{itemize}
\item nice --- приоритет процесса
\item pid --- уникальный идентификатор процесса
\item uid --- идентификатор пользователя, от имени которого запускается процесс
\item gid --- идентификатор группы, от имени которой запускается процесс
\item euid --- <<эффективный>> идентификатор пользователя
\item egid --- <<эффективный>> идентификатор группы
\item cgroupid --- задает множество ограничений на ресурсы
\item sessionid --- идентификатор сессии (нужен для массового завершения процессов)
\item pgid --- идентификатор процессной группы (используется для работы с терминалом)
\end{itemize}

Пример использования euid и egid:

passwd --- изменяет пароль. etc/shadow --- здесь хранится пароль. Мы запускаемся с gid = 0, uid =  0. Таким образом, euid и egid нужны для того, чтобы понять кому мы меняем пароль.

\begin{itemize}
\item getcwd --- позволяет получить текущий каталог
\item setcwd --- позволяет задать  текущий каталог
\item chroot --- позволяет сменить корень процесса
\end{itemize}

Ограничения на ресурсы, доступные процессу:
\begin{itemize}
\item Жесткие
\item Мягкие
\end{itemize}

\textbf{Ресурсы, которые могут быть доступны процессу:}
\begin{enumerate}
\item Размер виртуальной памяти
\item Размер стека
\item Количество открытых файлов
\item Размер файла core (файла дампа)
\item Размер pipe
\item Число создаваемых процессов
\end{enumerate}

\subsection*{Классификация ядер}

\begin{enumerate}
\item Монолитное ядро
\item Расширяемое ядро (реализуется с помощью модулей)

Модуль ядра --- некий объектный код, который компилируется отдельно от ядра, но обладает жестким интерфейсом вызовов и способом описания (имя модуля, зависимости).

\textit{insmode} --- подгружает модули в ОП

\item Микроядро
\begin{itemize}
\item Понятие процесса
\item Внутренняя шина
\end{itemize}
\end{enumerate}

Монолитное ядро работает быстрее, чем расширяемое. Но при этом сложно надежное написание, затруднена отладка.