\subsection{Память}

\begin{enumerate}
\item Оперативная память
\item Внешнее запоминающее устройство
\begin{itemize}
\item HDD
\begin{itemize}
\item Floppy
\end{itemize}
\item SSD
\item Оптика
\end{itemize}
\item Кэш процессора
\item Регистровая память
\end{enumerate}

\subsubsection{Время жизни информации}

\textbf{Тактовая частота} --- частота тактов.

\textbf{Такт} --- минимальное время, необходимое для того, чтобы фиксировать состояние процессора, чтобы получить гарантированное значение сигнала на всех входах и выходах.

\textbf{Генератор тактовой частоты} --- может дискретно изменять тактовую частоту.

Время жизни в регистровой памяти --- время, на протяжении которого процессор включен. Регистровая память работает со скоростью процессора.
\\

Два варианта:
\begin{itemize}
\item Кэш находится в процессоре. Работает медленнее регистровой памяти, время жизни --- время, на протяжении которого процессор включен.
\item Кэш --- некоторая микросхема, тесно связанная с процессором. Время жизни --- время, на протяжении которого процессор включен.
\end{itemize}

Выключая машину, оперативная память может продолжать существовать, при условии, что она сама потребляет электричество.
\\

\textbf{Характеристики ОП:}
\begin{itemize}
\item Частота шины
\item taccess --- время чтения
\item twrite --- время записи (время до освобождения шины)
\item trecycle --- время повторного обращения (время, после которого можно обратиться к той же ячейке, если до этого что-то записали в эту ячейку или извлекали из нее информацию).
\item trefresh --- время, через которое нужно повторно записывать в ячейку значение. (это необходимо, так как происходит утечка заряда из одной ячейки в другую, то есть 0 может стать 1 или 1 станет 0, таким образом, нужно обновить значение заряда).
\end{itemize}

\section{Процессор}

\textbf{Instruction pointer} --- регистр, задающий номер ячейки в ОП для извлечения оттуда слова, которое содержит некоторую команду.
\\

\textbf{Стадии работы процессора:}
\begin{enumerate}
\item Чтение инструкции (это делает устройство управления)
\item Декодирование (Decoder)
\item Выполнение инструкции АЛУ (при этом могут быть использованы дополнительные данные из ОП, а также регистры)
\item Сохранение результатов (в ОП и в регистрах)
\end{enumerate}

\subsection{Системы команд}

Команда:
\begin{itemize}
\item Код операции
\item Типы аргументов
\item Аргументы (для каждого выделено обычно либо 32 бит, либо 64 бит памяти, в зависимости от архитектуры процессора; иногда можно оптимизировать хранимые данные: под аргумент отводят меньшее количество бит)
\begin{itemize}
\item Число (интерпретируется либо как константа, либо как адрес в памяти; будем пока считать, что адрес задан номером ячейки)
\item Регистр
\end{itemize}
\end{itemize}

В \textit{microcode} задается реализация макрокоманд.
\\

\textbf{Системы команд:}
\begin{enumerate}
\item CISC(Complex Instruction Set Computing) --- команды могут иметь переменный размер.
\item RISC(Reduced Instruction Set Computing) --- пытается уменьшить размер команды. Размер команды почти всегда фиксирован. Команда \textit{prefetch} позволяет извлечь из памяти заданное количество команд и выполнять их потом параллельно.
\end{enumerate}

Классификация команд по количеству аргументов:
\begin{itemize}
\item Одноадресные. Операции происходят над единственным аргументом, результат записывается в регистр.
\item Двухадресные. Один из операндов используется для записи результата(обычно в первый).
\item Трехадресные. Результат записывается в третий операнд(вычисления происходят над первыми двумя).
\item Безадресные. Извлекают аргументы из памяти и кладут обратно в память. (Стековая машина)
\end{itemize}