\section{Вычислительная система}

\begin{enumerate}
\item Прикладное программное обеспечение;
\item Системное программное обеспечение. Граница с прикладным ПО довольно размыта;
\item Системы программирования (компиляторы, форматы представления данных\dots\newline Программы для написания программ и библиотеки);
\item Абстрактная машина (абстрактный вычислитель), вычислительные виртуальные ресурсы (например, файл, программа);
\item Управление физическими ресурсами;
\item Аппаратное обеспечение.

\end{enumerate}

Ядро, включая драйверы, лежит на уровнях 3-5, но 3-ий уровень захвачен не полностью. Драйверы предоставляют абстракцию, самостоятельно оперируя прерываниями и адресами.

Операционная система лежит на уровнях 2-5. Debian kFreeBSD является примером ОС, в которой на ядре FreeBSD предоставляется тот же системный интерфейс, что и в системах семейства Linux.

\subsection*{Механизм системных вызовов}
Есть всякие регистры, прерывания, assembler, но как это относится к программе на C? Если не использовать ничего дополнительного, то придётся под каждую конкретную машину переписывать код из-за разницы в этих интерфейсах. 

Для этого используют библиотеки. Есть библиотека libc, которая принудительно прилинковывается к каждой программе. Эта библиотека предоставляет интерфейс для взаимодействия с <<внутренностями>> операционной системы.

POSIX --- интерфейс системных вызовов, чья реализация содержится в libc. 
