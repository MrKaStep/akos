\subsubsection*{Запрет сброса TLB}

Некоторые страницы можно пометить \textbf{несбрасываемыми}

\subsection{Сегментно-страничная организация}

При такой организации виртуальный адрес в первую очередь проверяется \textbf{блоком сегментации}, после чего попадает в блок <<страничности>>, из которого идёт запрос к оперативной памяти. Каждый блок может по своему усмотрению изменить формат адреса.

\section{Работа с периферией}

\begin{itemize}
\item Порты ввода-вывода. Есть специальные команды:
\begin{verbatim}
in  <port>, <data>
out <data>, <port>
\end{verbatim}
\item Отображение адреса. Некоторая часть пространства виртуальных адресов выдаётся под внешние устройства. 

Например, в случае с видеокартой, при записи в виртуальный адрес, на самом деле происходит обращение к собственной памяти видеокарты посредством соответствующей шины. В реальности размер такого выделенного сегмента фиксирован, и при передаче большого массива данных приходится передавать данные по частям.
%TODO
%Также есть возможность отображения виртуального адреса на специальный диапазон оперативной памяти, 
\item Асинхронный режим. Реализуется при помощи DMA (Direct Memory Access). При асинхронном режиме есть отображение с виртуальной памяти на память внешнего устройства, а также внешнее устройство связано с оперативной памятью. Механизм общения выглядит так:
\begin{enumerate}
\item Модификация таблицы страниц по защите диапазона адресов оперативной памяти.
\item Формирование команды внешнему устройству. В отображаемые адреса кладём следующую информацию:
\begin{itemize}
\item Номер прерывания
\item Код операции
\item Диапазон адресов физической памяти
\end{itemize}
\item Возврат процессора к нормальному режиму
\item Прерывание. Чтение данных из диапазона отображаемых адресов (информация о том, как завершилась операция). Выполняется обработчиком прерывания.
\item Переписывание данных из <<DMA>>-диапазона в структуры ядра. При этом обработчик прерывания должен понимать, куда нужно положить эти данные (например, данные с жёсткого диска и из сети нужно класть в соответствующие буферы).
\item Изменение таблиц страниц с целью снятия блокировки с виртуального адреса. Этот шаг необязательно выполняется, например при изначальной инициализации <<DMA>> диапазона.
\end{enumerate}
\end{itemize}

\section{Прерывания и их обработка}

Все прерывания пронумерованы

\begin{itemize}
\item Регистр прерываний. При прерывании регистр выставляется в соответствующее значение.
\item Регистр маски прерывания. При выставлении нужных битов в 1 можно замаскировать прерывание и процессор не будет обращать на него внимания
\item Вектор прерываний. Изначально предоставляется BIOS, после чего переписывается ОС и используется для своих нужд. Ссылка на физическую память этого вектора лежит в регистре \verb!idtr!. Это некоторая таблица, содержащая адреса входов в обработчики прерываний.
\end{itemize}

У прерываний есть приоритет. Обычно, чем больше номер прерывания, тем ниже у него приоритет.

\subsubsection*{APIC}

%TODO

На материнской плате находится контроллер прерываний. На каждом процессоре есть LPIC --- локальный контроллер прерываний. При поступлении сигнала с внешнего устройства на контроллер прерываний прерывание по системе round-robin передаётся процессорам. После был создана система MSI, затем MSIX.

\subsection{Передача прерывания на процессор}
После передачи прерывания процессору (в LPIC) и понимания, что данное прерывание не замаскировано, происходит следующее:

\begin{enumerate}
\item Аппаратная обработка прерывания:
\begin{enumerate}
\item Сохраняем в стек \verb!RIP! (место выполнения), \verb!FLAGS!, регистр маски, сегментные регистры;
\item Регистр маски выставляется в 1 (все возможные прерывания маскируются);
\item Вызываем обработчик прерывания с нулевым уровнем привилегий (должен быть обеспечен ядром ОС).
\end{enumerate}
\item Программная обработка прерывания:
\begin{enumerate}
\item Складываем в стек регистры, которыми пользуемся;
\item Производим <<срочные действия>> (например, если ранее пришло прерывание от таймера);
\item Размаскируем прерывания, которые не будут мешать;
\item Длительная обработка прерывания;
\item Восстановление регистров;
\item Вызов инструкции \verb!iret! --- возврат в нормальный режим работы (или в предыдущий обработчик).
\end{enumerate}
\end{enumerate}

\subsection{Создание прерываний}
Для создания прерывания есть инструкция \verb!int! (interrupt)
\begin{verbatim}
int <interruption #>
\end{verbatim}

Прерывания делятся на
\begin{itemize}
\item Внутренние прерывания;
\item Внешние прерывания.
\end{itemize}

Внутренние прерывания обычно более приоритетны. 

Примеры внутренних прерываний:
\begin{itemize}
\item Деление на 0
\item page fault
\item access error
\item trap --- необходимость отладки
\item illegal insruction
\item bad memory access
\item \ldots
\end{itemize}

\subsection{Прерывание таймера}

\subsubsection{Watch dog}

В процессоре есть регистр \verb!NMI!. Со временем число в этом регистре инкрементируется или декрементируется. В тот момент, когда этот регистр стал нулём, создаётся немаскируемое прерывание. Если это случится, мы получим BSOD. Если получится, будет создан дамп памяти. Это нужно для отладки операционной системы. 

Такой же таймер есть на материнской плате. Он используется для отладки внешних устройств. Если он занулится, то будет перезагрузка питанием. 









