\subsection{Уровень аппаратного обеспечения}

\begin{itemize}
\item Материнская плата
\begin{itemize}
\item разъём шины питания;
\item разъёмы оперативной памяти;
\item разъём(ы) под процессор(ы);
\item разъёмы расширений PCI-E (Peripheral component interconnect - Express). Раньше чаще использовались другие разъёмы: AGP (Accelerated Graphics Port), PCI, ISA (Industry Standard Architecture). Шина ISA была аналоговой, т.е. сигнал не дискретизировался. Остальные же шины уже были цифровые.
\item разъёмы для <<медленных устройств>>. Это разъёмы, к которым подключаются устройства, связанные с человеком. Сейчас большинство <<медленных устройств>> переехали на
\item USB (Universal Serial Bus).
\item SATA (Serial ATA) --- последовательный интерфейс обмена данными с накопителями, SCSI <<скази>> (Small Computers System Interface) --- интерфейс для общения компьютера с периферийными устройствами.
\item BIOS (Basic Input-Output System). Эта микросхема подключена ко всем шинам. 
\item батарейка, которая позволяет сохранять астрономическое время, например, для запуска компьютера в определённое время.
\end{itemize}
На материнской плате есть набор шин:
\begin{itemize}
\item Процессор - RAM (<<северный мост>>).
\item RAM - контроллер периферии.
\item Процессор - контроллер периферии (<<Южный мост>>, обычно PCI-E). Остальные шины являются абонентами этой шины.
\end{itemize}
Эти три шины называются быстрыми шинами, так как контроллер PCI-E. В системе Linux при помощи команд \verb!lspci! и \verb!lsusb! можно посмотреть набор устройств, подключённых к соответствующей шине.

Компьютер суть есть иерархия шин, к которым подключаются все устройства. 
\end{itemize}

\subsection*{Архитектура фон Неймана}

Компьютер фон Неймана --- исполнитель некоторой программы, состоящей из команд.

\begin{itemize}
\item Центральный процессор
\begin{itemize}
\item АЛУ (Арифметико-логическое устройство)
\item УУ (устройство управления)
\end{itemize}
\item Оперативная память, подключённая к процессору;
\item Внешние устройства, подключённые к процессору.
\end{itemize}

В компьютере фон Неймана есть понятие \textbf{такт} --- время, за которое выполняется одна команда. Команды выбираются из оперативной памяти в некотором порядке.\\

\underline{Принципы фон Неймана:}
\begin{enumerate}
\item Принцип хранимой программы (программа хранится в оперативной памяти).
\item Принцип адресности. Оперативная память разбита на набор ячеек, одна ячейка называется \textbf{словом}.
\item Принцип двоичного кодирования.
\item Принцип однородности. Команды и данные хранятся в одной и той же памяти и внешне в памяти неразличимы.
\end{enumerate}

Эта архитектура небезопасна, так как программа может затереть абсолютно всё, включая коды всех программ.

Сейчас для защиты памяти используют механизм виртуальной памяти и тегированную память. Тегированная память отличается от обычной тем, что слово представляет из себя не только адрес, но и некоторой тег, который предоставляет некоторую служебную информацию.

\subsection*{Гарвардская архитектура}

Эта архитектура отличается от фон Неймановской тем, что инструкции и память физически разделены на разные устройства хранения и каналы к этим устройствам.\\
Она не используется, так как никак нельзя загрузить новую программу, но зато она безопасна, в отличие от фон Неймановской.