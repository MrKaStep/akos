\section{Файловая система}

\subsection{Где может храниться файловая система}

Файловая система может храниться на \textbf{блочном устройстве}. Это устройство, у которого определены операции чтения и записи, но обратиться к одному байту нельзя, а можно читать и записывать только \textbf{блоками}.

Блочные устройства могут быть физическими и виртуальными. 

Физические:
\begin{itemize}
\item HDD
\item SSD
\item Оптика (Blu-ray, DVD, ...)
\item Магнитная лента
\end{itemize}

Все устройства делятся на устройства
\begin{itemize}
\item Последовательного доступа --- к ним относятся магнитные ленты и, в некоторой степени, оптические диски.

Чтобы добраться до нужной ячейки, нужно пролистать всё до этой самой ячейки.
\item Произвольного доступа
\end{itemize}

\subsubsection{Магнитная лента}

Лента устроена как, собственно, лента и головка записи-чтения. Лента делится на дорожки, которые делятся на биты. У ленты есть свойство повреждаться, поэтому нужен достаточно надежный способ записи.

Надёжность зависит от плотности записи, которая зависит от таких характеристик как зерно лента. Для надежности некоторые паттерны записываются несколько раз.

Файловая система чаще хранится на паре лента+SSD, где на SSD хранится оглавление того, что лежит на ленте.

\subsubsection{Оптика}

Диск состоит из круговых дорожек и радиальных секторов.

Проблемы:

\begin{enumerate}
\item Одноразовая запись
\item Перезапись конечное число раз, много меньшее, чем у SSD
\end{enumerate}

Файловая система для оптических дисков --- ISO 9660. Эта файловая система похожа на систему для ленты. В этой файловой системе есть так называемые сессии. В начале диска хранится оглавление и копия оглавления. Остальное хранится как файлы и директории на диске, причём они не бьются на блоки. 

В случае необходимости удаления файл не стирается, а просто <<забывается>>, а при изменении просто записывается новая копия файла, снова забывая про старую.

Есть другая файловая система --- UDF. Эта система блочная. Файлы записываются по блокам, система предназначена для изменения файлов и возможности перезаписи. В этой системе файл не копируется полностью, а обновляется только небольшое число блоков.

\subsubsection{SSD}

SSD имеет ограниченное количество циклов перезаписи. Для решения этой проблемы при последовательной записи блоки перемешиваются как можно более случайно для равномерного износа.

Что же хорошо и плохо хранить на SSD? 

\begin{itemize}
\item  \textbf{Плохо}. swap-раздел. Казалось бы, всё будет хорошо, веь большая скорость --- идеально для swap'а. Но при этом нужно быть готовым, что устройство быстро выйдет из строя.

\item Хорошо --- корень и \verb!/usr!, так как они часто читаются и редко записываются

\end{itemize}

\subsubsection{HDD}

Жёсткий диск устроен как набор поверхностей, над которыми висят прикреплённые к общему кронштейну головки чтения-записи. У данных на жёстком диске есть три координаты (поверхность, дорожка, сектор). Понятно, как определить поверхность и дорожку, а как определить сектор?

Сектор определяется при помощи отсчёта нужного количества миллисекунд.

Плохо в жестких дисках то, что перемещение головки занимает много времени, а быстрее перемещать нельзя из-за проблемы перепада температур. 

Жёсткий диск сам оптимизирует количество перемещений головки, раньше этим занималась операционная система.

Для более простого с точки зрения ОС доступа была придумана такая вещь, как LBA (Linear Block Array). \\
\\
\\
\subsection*{Виртуальные блочные устройства}

\subsubsection{Таблицы разделов} 

В этом случае какой-то участок LBA-устройства выделяется на хранение таблицы разделов, а остальное пространство делится на части, которые и называются разделами. Далее предполагается, что на этих разделах будет находиться файловая система.

Множество разделов для файловой системы полезно тем, что позволяет локализовать урон при bad block, а также 

В мире Intel есть две основные системы:
\begin{itemize}
\item MBR (Master Boot Record) --- поддерживает лишь четыре раздела. Нельзя сделать MBR больше, чем на 1 TB.

extended раздел --- раздел со своей собственной MBR.

Устроена она так:

\begin{itemize}
\item 512 байт загрузчика
\item MBR
\item MBR (копия)
\item Разделы. Для хранения типа раздела используют 1 байт.
\end{itemize}
\item GPT (GUID Partition Table). Поддерживает вложенные extended-разделы, копия хранится в конце, а не в начале. Для хранения типа раздела используют уже 2 байта. Есть два важных типа разделов: bios boot data и efi boot.

Также у каждого раздела есть подпись, которая как-то обеспе
\end{itemize}

\subsubsection{Сетевые блочные устройства}

\begin{verbatim}
/dev/nfs
/dev/nvrom...
\end{verbatim}

\subsubsection{Блочные устройства в оперативной памяти}

Минус --- при выключении питания теряются данные

\subsubsection{Много дисков как одно устройство}

Есть два главных типа таких устройств:

\begin{itemize}
\item RAID 0-6. Остальные номера --- комбинации. В RAID есть определение <<полосы>>
\begin{itemize}
\item RAID 0. Создаётся устройство с размером --- суммой размеров входящих в него устройств. При этом появляется явление расслоения памяти, когда одна полоса делится на части и части записываются на разные диски (например, в порядке 1, 2, 3, 1, 2,...). В этом случае скорость записи-чтения теоретически может увеличиться в n раз. Проблема в том, что если один из дисков выйдет из строя, то всё устройство прекращает своё существование. Удобно для создания каких-нибудь tmp

\item RAID 1. Предполагает два диска. Размер совпадает с размером каждого из дисков в массиве. Запись происходит сразу на два диска, причём при поломке одного из дисков система происходит перевод массива в режим read-only или происходит подключение spare-диска, информация при этом не теряется.

\item RAID 2. Содержит n дисков. Информация как-то более-менее равномерно размазана по всем дискам. Информация хранится в виде массива и кодами Хэмминга, восстанавливающими ошибки. 

\item RAID 3. Один из дисков используется как диск с контрольными суммами (блок чётности), где хранятся XOR-суммы данных на остальных дисках. Так, при помощи контрольных сумм можно будет восстановить любой повреждённый диск. Проблема в неравномерной нагрузке на диски.

\item RAID 4. RAID 4, но деление полосы на байты, а не на блоки.

\item RAID 5. RAID 3, но контрольные суммы распределены по дискам.

\item RAID 6. RAID 5, но 2 контрольные суммы. Таким образом гарантируется восстановление после выхода из строя двух жёстких дисков.
\end{itemize}

\item LVM. Создаётся виртуальное блочное устройство Logical Volume. Он состоит из участков других блочных устройств. Также есть такая вещь, как Volume group, которая может состоять из нескольких Logical Volume и множества блочных устройств. В дальнейшем блочные устройства можно распределять по Logical Volume.

Эта система предназначена для сжатия и расширения файловой системы, когда это необходимо. Например, в больших дата-центрах могут использовать LVM натянутый на RAID 6.
\end{itemize}

\section{Процесс загрузки}

\begin{enumerate}
\item Аппаратная стадия
\item Работа BIOS
\item Загрузчик 1-й стадии
\item Загрузчик 2-й стадии
\item Старт ядра
\item Старт init
\item Разворачивание служб и пользовательских программ
\end{enumerate}


\setcounter{subsection}{-1}

\subsection{Аппаратная стадия}

Обычно в эту стадию не задействуется ЦП, эта стадия --- прерогатива материнской платы. Цель этой стадии --- проверка того, что подключено к материнской плате (наличие процессора и памяти. проверка памяти). Затем настраивается контроллер прерываний, чтобы BIOS мог работать с ними.

\subsection{Работа BIOS}

BIOS содержит в себе некоторые настройки. BIOS грузится в оперативную память, используется ЦП. Грузятся некоторые настройки, например, производится опрос устройств, которые были записаны в системный раздел памяти, для поиска загрузчика 1-й стадии. Или переводит процессор в режим наибольшей совместимости

\subsection{Загрузчик 1-й стадии}

Запускает загрузчик 2-й стадии.

\subsection{Загрузчик 2-й стадии}

Перевод процессора обратно в нужный режим. Обычно принимает что-то в формате файловых систем, что передавать ядру и как его размещать в оперативной памяти.

Также он может искать драйвера в файловой системе.

После размещения ядра, управление передаётся ему и загрузчик перестаёт существовать

\subsection{Старт ядра}

Разворачиваются все необходимые таблицы, распознаёт и конфигурирует устройства так, как ему надо. Затем монтируется корневая файловая системаи запускается init.

\subsection{Старт init}

В случае Linux есть systemd и sysvinit. 

\subsection{Разворачивание служб и пользовательских программ}