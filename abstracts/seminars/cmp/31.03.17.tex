\section{Демонизация}



В UNIX демон --- это фоновый процесс, не взаимодействующий с терминалом и создающий новую сессию. Демоны запускаются процессом \verb!init! с root-правами. Но это считается небезопасным, поэтому он должен сменить свои права на права другого, возможно, реального пользователя.

Есть команда \verb!getpwuid!.

Сообщения о каких-то событиях записываются в специализированные файлы журнала. Потоки stdin и stdout обычно закрыты или перенаправлены.

Для смены корня можно использовать \verb!chroot!. 

Иногда нам хочется, чтобы одновременно был запущен лишь один демон соответствующего типа. Для этого обычно используется каталог \verb!/var/run/!, в котором создаётся соответствующий классу демонов файл, который блокируется на время жизни демона.

По умолчанию считается, что демоны пишут свою информацию в syslog. Работа с ним начинается с вызова функции \verb!openlog!. Для вывода используется
\begin{verbatim}
syslog(<log level>, <format string>, ...);
\end{verbatim}

\subsection{Как стать демоном?}

У каждого процесса есть идентификатор сессии \verb!sid! и идентификатор группы процессов \verb!pgid!. Чтобы стать демоном нужно задать новый идентификатор процесса и новую процессную группу. И то, и другое можно создать только из собственного \verb!pid'a!. но перед этим нужно закрыть все свои терминалы.

\begin{verbatim}
getsid
setsid
getpgid
setpgid
\end{verbatim}

Вообще у каждого процесса есть
\begin{verbatim}
pid
ppid
sid
pgid
uid
gid
euid --- effective
egid --- effective
\end{verbatim}

Но вернёмся к нашим баранам, а точнее --- демонам. Как мы можем создать демона?

\begin{enumerate}
\item Закрыть стандартные потоки 0-2 при помощи \verb!close!.
\item \verb!fork!'нуться. А лучше дважды, чтобы исходное окружение нас вообще не трогало. Ну да ладно, один раз.
\item Меняем процессную группу.
\item Меняем сессию.
\item Меняем как надо \verb!uid, gid, euid, egid!. Ведь обычно мы запускаемся \verb!init!'ом, с root-правами, а это опасно.
\end{enumerate}